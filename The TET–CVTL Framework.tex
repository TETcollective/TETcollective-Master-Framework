\documentclass[11pt,a4paper]{article}
\usepackage[margin=1in]{geometry}
\usepackage{amsmath}
\usepackage{amssymb}
\usepackage{hyperref}
\usepackage{graphicx}
\usepackage{titling}
\usepackage{fancyhdr}
\usepackage{lastpage}

\hypersetup{
    colorlinks=true,
    linkcolor=blue,
    citecolor=blue,
    urlcolor=blue
}

\pagestyle{fancy}
\fancyhf{}
\rhead{Simon Soliman – TETcollective}
\lhead{The TET–CVTL Framework}
\cfoot{Page \thepage\ of \pageref{LastPage}}

\title{The TET–CVTL Framework: From Local Vacuum Torque Probe to Supersymmetric Klein-Bottle Universe – Comprehensive Theory and Experimental Implications}

\author{Simon Soliman \\
Independent Researcher – Rome, Italy \\
\texttt{mechudzu@outlook.it} | ORCID: 0009-0002-3533-3772 \\
TETcollective – Topology \& Entanglement Theory}

\date{December 2025}

\begin{document}

\maketitle

\begin{center}
    \textit{This work is licensed under a Creative Commons Attribution-NonCommercial 4.0 International License (CC BY-NC 4.0).\\
    Commercial use requires explicit permission from the author.\\
    Contact: tetcollective@proton.me | \url{https://creativecommons.org/licenses/by-nc/4.0/}}
\end{center}

\vspace{1cm}

\begin{abstract}
This comprehensive paper presents the unified TET–CVTL framework, integrating the experimental vacuum torque probe (TET–CVTL device) with the theoretical model of a supersymmetric non-orientable de Sitter universe with Klein bottle topology. The primordial knot cycles are compactified on the Riemann sphere $\mathbb{CP}^1$, providing closure of zero and infinity via the inversion map $1/z$. The 3-6-9 topological-algebraic invariant is rigorously derived from Euler's identity applied to anyonic phases in the non-orientable geometry. The framework yields falsifiable predictions and a consistent basis for local vacuum energy extraction. All results build on previous works listed in the references.
\end{abstract}

\section{Introduction}

The TET–CVTL framework originated from the observation of magnetic entanglement in common materials and evolved into a unified multiscale description of fundamental forces. The device TET–CVTL serves as a local laboratory probe, while the theoretical model describes the global cosmology as a supersymmetric Klein-bottle universe compactified on the Riemann sphere.

This paper integrates the progressive developments from TET–CVTL v6.1 through the General Theory v2.0, providing a complete and self-consistent formulation.

\section{The TET–CVTL Device}

The TET–CVTL device induces controlled gradients in vacuum entanglement density through phased spiral configurations, enabling measurable reactionless thrust and torque from the zero-point field \cite{tet-cvtl-v61}. Subsequent refinements (v6.2--v6.5) improved bounds on anyonic phase catalysis and vacuum amplification \cite{tet-cvtl-updates}.

Laboratory predictions include thrust in the range 1--12 kN at low power input (<5 W), subject to experimental verification in the 2026--2029 timeframe.

\section{Topological Foundation}

The observable universe is the interior of a supersymmetric non-orientable de Sitter black hole with global topology
\begin{equation}
M = K_2 \times CY_3 \times G_2,
\end{equation}
where the spatial section is rendered non-orientable by a Z$_2$ orbifold action \cite{general-theory-v1, general-theory-v2}. Primordial knots parameterize vacuum entanglement across scales.

\section{Riemann Sphere Compactification}

The complex phase plane of primordial knots is compactified on the Riemann sphere $\mathbb{CP}^1$ via stereographic projection. The inversion map $f(z) = 1/z$ exchanges zero (primordial singularity) and infinity (cosmological horizon), preserving the fundamental group $\pi_1(K_2)$ through anti-holomorphic isometry \cite{general-theory-v2}.

\section{Euler Closure and the 3-6-9 Invariant}

The non-orientable twist enforces $\theta = \pi$, yielding Euler's identity
\begin{equation}
e^{i\pi} + 1 = 0.
\end{equation}
This generates fermionic statistics, 6 real supercharges from Pin$^+$ supersymmetric doubling, and 9 from E$_8$ self-duality, establishing 3-6-9 as the unique topological-algebraic invariant.

\section{Observational and Laboratory Predictions}

The framework provides falsifiable predictions:
\begin{itemize}
\item Azimuthal 3-fold modulation in CMB entanglement entropy fluctuations.
\item Gravitational-wave echo resonance ratios 3:6:9 from pole clustering on $\mathbb{CP}^1$.
\item Suppression of dark matter signal in perfect superconductors via Meißner effect on vacuum magnetic entanglement flux.
\item Measurable reactionless thrust from TET–CVTL device in controlled laboratory conditions.
\end{itemize}

Experimental tests are scheduled for 2026--2029.

\section{Implications for Vacuum Energy Extraction}

Global recycling of vacuum torque is unitary ($\eta = 1$), but cosmologically dilute. Local extraction via TET–CVTL enables practical applications, including:
\begin{itemize}
\item Reactionless propulsion without propellant expenditure, enabling economic interplanetary and potentially interstellar travel.
\item Scalable reactionless generators for clean, unlimited energy production.
\item Potential resolution of global energy constraints.
\end{itemize}

All practical implications are subject to experimental confirmation of the predicted performance.

\subsection{Commercial and Collaborative Implications}

The author reserves commercial rights under the CC BY-NC 4.0 license. Academic and non-commercial use is encouraged. Organizations interested in commercial applications, licensing, or collaborative development of the TET–CVTL technology or related vacuum engineering are invited to contact the author at tetcollective@proton.me.

\section{Conclusion and Future Work}

The TET–CVTL framework provides a unified description from local vacuum probe to global cosmology. Future work includes experimental validation of laboratory predictions and refinement of the topological bootstrap mechanism.

\begin{thebibliography}{9}

\bibitem{tet-cvtl-v61}
Soliman, S. (2025). TET–CVTL v6.1. \href{https://doi.org/10.5281/zenodo.17843778}{DOI: 10.5281/zenodo.17843778}

\bibitem{tet-cvtl-updates}
Soliman, S. (2025). Subsequent TET–CVTL refinements. DOIs: 10.5281/zenodo.17853424, 10.5281/zenodo.17857154, 10.5281/zenodo.17864102, 10.5281/zenodo.17870523, 10.5281/zenodo.17881930

\bibitem{general-theory-v1}
Soliman, S. (2025). The Supersymmetric Klein-Bottle Universe v1.0. \href{https://doi.org/10.5281/zenodo.17844202}{DOI: 10.5281/zenodo.17844202}

\bibitem{general-theory-v2}
Soliman, S. (2025). The Supersymmetric Klein-Bottle Universe v2.0. \href{https://doi.org/10.5281/zenodo.18043224}{DOI: 10.5281/zenodo.18043224}

\end{thebibliography}

\section*{License and Relation to Previous Works}

This comprehensive work integrates and significantly refines previous publications listed in the references.

The new calculations, derivations, theoretical developments, and practical implications presented in this document are licensed under a \textbf{Creative Commons Attribution-NonCommercial 4.0 International License (CC BY-NC 4.0)}.

Previous versions of the framework and related works remain under their original licenses as specified in their respective Zenodo records.

Commercial use of the material in this document requires explicit permission from the author. Interested parties are invited to contact: \texttt{tetcollective@proton.me}.

Full license text: \url{https://creativecommons.org/licenses/by-nc/4.0/}


\begin{center}
    \vspace{2cm}
    \large
    This document is licensed under \\
    \textbf{Creative Commons Attribution-NonCommercial 4.0 International (CC BY-NC 4.0)} \\
    \url{https://creativecommons.org/licenses/by-nc/4.0/} \\
    \\
    Commercial inquiries: tetcollective@proton.me
\end{center}

\end{document}